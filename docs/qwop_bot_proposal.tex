\def\year{2021}\relax
%File: formatting-instructions-latex-2021.tex
%release 2021.2
\documentclass[letterpaper]{article} % DO NOT CHANGE THIS
\usepackage{aaai21}  % DO NOT CHANGE THIS
\usepackage{times}  % DO NOT CHANGE THIS
\usepackage{helvet} % DO NOT CHANGE THIS
\usepackage{courier}  % DO NOT CHANGE THIS
\usepackage[hyphens]{url}  % DO NOT CHANGE THIS
\usepackage{graphicx} % DO NOT CHANGE THIS
\urlstyle{rm} % DO NOT CHANGE THIS
\def\UrlFont{\rm}  % DO NOT CHANGE THIS
\usepackage{natbib}  % DO NOT CHANGE THIS AND DO NOT ADD ANY OPTIONS TO IT
\usepackage{caption} % DO NOT CHANGE THIS AND DO NOT ADD ANY OPTIONS TO IT
\frenchspacing  % DO NOT CHANGE THIS
\setlength{\pdfpagewidth}{8.5in}  % DO NOT CHANGE THIS
\setlength{\pdfpageheight}{11in}  % DO NOT CHANGE THIS
%\nocopyright
%PDF Info Is REQUIRED.
% For /Author, add all authors within the parentheses, separated by commas. No accents or commands.
% For /Title, add Title in Mixed Case. No accents or commands. Retain the parentheses.
\pdfinfo{
/Title (AAAI Press Formatting Instructions for Authors Using LaTeX -- A Guide)
/Author (AAAI Press Staff, Pater Patel Schneider, Sunil Issar, J. Scott Penberthy, George Ferguson, Hans Guesgen, Francisco Cruz, Marc Pujol-Gonzalez)
/TemplateVersion (2021.2)
} %Leave this
% /Title ()
% Put your actual complete title (no codes, scripts, shortcuts, or LaTeX commands) within the parentheses in mixed case
% Leave the space between \Title and the beginning parenthesis alone
% /Author ()
% Put your actual complete list of authors (no codes, scripts, shortcuts, or LaTeX commands) within the parentheses in mixed case.
% Each author should be only by a comma. If the name contains accents, remove them. If there are any LaTeX commands,
% remove them.

% DISALLOWED PACKAGES
% \usepackage{authblk} -- This package is specifically forbidden
% \usepackage{balance} -- This package is specifically forbidden
% \usepackage{color (if used in text)
% \usepackage{CJK} -- This package is specifically forbidden
% \usepackage{float} -- This package is specifically forbidden
% \usepackage{flushend} -- This package is specifically forbidden
% \usepackage{fontenc} -- This package is specifically forbidden
% \usepackage{fullpage} -- This package is specifically forbidden
% \usepackage{geometry} -- This package is specifically forbidden
% \usepackage{grffile} -- This package is specifically forbidden
% \usepackage{hyperref} -- This package is specifically forbidden
% \usepackage{navigator} -- This package is specifically forbidden
% (or any other package that embeds links such as navigator or hyperref)
% \indentfirst} -- This package is specifically forbidden
% \layout} -- This package is specifically forbidden
% \multicol} -- This package is specifically forbidden
% \nameref} -- This package is specifically forbidden
% \usepackage{savetrees} -- This package is specifically forbidden
% \usepackage{setspace} -- This package is specifically forbidden
% \usepackage{stfloats} -- This package is specifically forbidden
% \usepackage{tabu} -- This package is specifically forbidden
% \usepackage{titlesec} -- This package is specifically forbidden
% \usepackage{tocbibind} -- This package is specifically forbidden
% \usepackage{ulem} -- This package is specifically forbidden
% \usepackage{wrapfig} -- This package is specifically forbidden
% DISALLOWED COMMANDS
% \nocopyright -- Your paper will not be published if you use this command
% \addtolength -- This command may not be used
% \balance -- This command may not be used
% \baselinestretch -- Your paper will not be published if you use this command
% \clearpage -- No page breaks of any kind may be used for the final version of your paper
% \columnsep -- This command may not be used
% \newpage -- No page breaks of any kind may be used for the final version of your paper
% \pagebreak -- No page breaks of any kind may be used for the final version of your paperr
% \pagestyle -- This command may not be used
% \tiny -- This is not an acceptable font size.
% \vspace{- -- No negative value may be used in proximity of a caption, figure, table, section, subsection, subsubsection, or reference
% \vskip{- -- No negative value may be used to alter spacing above or below a caption, figure, table, section, subsection, subsubsection, or reference

\setcounter{secnumdepth}{0} %May be changed to 1 or 2 if section numbers are desired.

% The file aaai21.sty is the style file for AAAI Press
% proceedings, working notes, and technical reports.
%

% Title

% Your title must be in mixed case, not sentence case.
% That means all verbs (including short verbs like be, is, using,and go),
% nouns, adverbs, adjectives should be capitalized, including both words in hyphenated terms, while
% articles, conjunctions, and prepositions are lower case unless they
% directly follow a colon or long dash

\title{Walk To Run: Teaching An  Intelligent Agent To Play QWOP }
\author{
    %Authors
    % All authors must be in the same font size and format.
    Written by Sonny Smith, Matthew Oros, Michael Terekhov
    \\
}
\affiliations{
    %Afiliations
   
    %If you have multiple authors and multiple affiliations
    % use superscripts in text and roman font to identify them.
    %For example,

    % Sunil Issar, \textsuperscript{\rm 2}
    % J. Scott Penberthy, \textsuperscript{\rm 3}
    % George Ferguson,\textsuperscript{\rm 4}
    % Hans Guesgen, \textsuperscript{\rm 5}.
    % Note that the comma should be placed BEFORE the superscript for optimum readability

    275 Eastland Road\\
    Berea, Ohio 44017\\
    % email address must be in roman text type, not monospace or sans serif

    % See more examples next
}
\iffalse
%Example, Single Author, ->> remove \iffalse,\fi and place them surrounding AAAI title to use it
\title{My Publication Title --- Single Author}
\author {
    % Author
    Author Name \\
}

\affiliations{
    Affiliation \\
    Affiliation Line 2 \\
    name@example.com
}
\fi

\iffalse
%Example, Multiple Authors, ->> remove \iffalse,\fi and place them surrounding AAAI title to use it
\title{My Publication Title --- Multiple Authors}
\author {
    % Authors
    First Author Name,\textsuperscript{\rm 1}
    Second Author Name, \textsuperscript{\rm 2}
    Third Author Name \textsuperscript{\rm 1} \\
}
\affiliations {
    % Affiliations
    \textsuperscript{\rm 1} Affiliation 1 \\
    \textsuperscript{\rm 2} Affiliation 2 \\
    firstAuthor@affiliation1.com, secondAuthor@affilation2.com, thirdAuthor@affiliation1.com
}
\fi
\begin{document}

\maketitle

\begin{abstract}
We have decided to recreate the game QWOP and create an AI that can sufficiently play the game. In QWOP, a player attempts to control a track runner’s thighs and legs using Q and W and calves using O and P to make them run without falling over. Our goal is to create an AI that can play this game and improve at it to eventually be able to travel 100m. We intend to use a combination of approaches that includes a genetic algorithm, a neural network, and reinforcement learning to train the AI to improve its gameplay. Our plan is to develop the AI using milestones such as crawling, standing, walking, etc. We intend to further expand this plan with more specific algorithms, milestones, etc.
\end{abstract}

\noindent The core issue that this program intends to solve is creating a biomechanical model of locomotion. This issue is bioinspired because most land-borne animals must learn to balance and walk on their own. This problem is explored in many fields including engineering, medicine, and robotics. Our goal is to successfully develop a program that can learn to maneuver the character without falling over.

Our approach to the problem is to develop a neural network and use a genetic algorithm and reinforcement learning to train it.

The neural network will be given a set of inputs, which involves the positions of all the limbs in the environment. The goal is to create a deep enough neural network with enough layers so we can train it by merging the best performing networks.

To optimize the weights of the neural network we will use a genetic algorithm. The approach that we will use is population-based. In such an approach, we generate multiple agents with various neural networks, which have randomly assigned weights and biases for the first generation and as they evolve the weights will change based on the successful performances of the networks.

To further improve the agent's performance we will use reinforcement learning. For this portion of the development, we will develop a utility function that would indicate how well the agent performs to make further updates to the neural network's weights based on the rewards received by the agent during run-time. The simplest version of the utility function that we have in mind is to give rewards if the figure moves forward and penalize it for falling or moving backward to update the weights in real-time. We will potentially use a replay buffer to store the outputs of the utility function (agent's experiences) for further improvements in the stability of the learning process.

To determine the effectiveness of the solution, objective quantitative measurements will be taken such as the the distance the figure has traversed. To determine effectiveness of the training solution, there would have to be evidence of an improvement in overall performance in movement from the first generation to multiple later generations. Even if the original goal is not met, any improvements indicates effectiveness and thus some success.

We believe that we will be successful because we have planned out a system of a neural network and genetic algorithm/reinforcement learning to help accomplish our goal. Our neural network has been coded to fully perceive the character's body and movements relative to the environment. This will allow us to determine if the character is balancing and measure its movement. We already have a functional draft of a neural network that can be used and upon which we can expand. We believe that our genetic algorithm/reinforcement learning approach will allow our program to learn the best approach to balancing and walking. Due to the genetic algorithm, we would be able to assign starting weights that would have some meaning due to trial and error. Due to reinforcement learning the weights and the actions would be polished (further improved). We believe that such a combination will give us better results compared to the ones that would be given if we were only to use one of the mentioned above algorithms.

If the original goal of having the figure traverse 100 meters is too ambitious for the time constraint given or complexity of the training solution, then smaller goals can be assigned such crawling, balance, or jumping. Even if only these smaller goals are achieved then it would still be evidence of the effectiveness of our learning solution. This would further increase the likelihood of producing a meaningful result even if the original goal is not achieved.

\end{document}